\documentclass[
  aspectratio=169,
  professionalfonts
]{beamer}

% Theme
%\usetheme{Madrid}
%\usecolortheme{default}

% Packages
\usepackage[utf8]{inputenc}
\usepackage[T1]{fontenc}
\usepackage{lmodern}
\usepackage{amsmath, amssymb}
\usepackage{graphicx}
\usepackage{booktabs}
\usepackage{comment}

% Metadata
\title[Short Title]{Full Presentation Title}
\author[Name]{Your Name}
\institute[Institute]{Your Institute}
\date{\today}

% Options
\setbeamertemplate{footline}[frame number]
\setbeamertemplate{itemize item}{\textbullet}
\setbeamertemplate{itemize subitem}{--}
\setbeamertemplate{itemize subsubitem}{$\star$}


\begin{document}

\maketitle

% -----------

\begin{frame}
  \titlepage
\end{frame}

% -----------

\begin{frame}{Outline}
  \tableofcontents
\end{frame}

% -----------

\section{Introduction}\label{s:intro}

\begin{frame}{Motivation}
\begin{itemize}
  \item Why this topic matters \pause
  \item Key challenge
  \item Goal of this talk
  \item<3> Content
  \item<2-> Stuff
  \item<-2>Hallo
\end{itemize}
Referenz \nameref{s:intro}
\end{frame}

\begin{frame}{Beispiel}
\alert<2>{highlighted text}
\end{frame}


\begin{frame}{Beispiel}
Erste Zeile

\onslide<2->{Zweite Zeile}

Dritte Zeile
\end{frame}

\begin{frame}{Beispiel}
Erste Zeile

\only<2>{Zweite Zeile}
%Only nur zum Ersetzen verwenden
\only<3>{Dritte Zeile}
\end{frame}

\begin{comment}
\begin{frame}{Titel, aberkeinUntertitel!}
\onslide<3->{Taucht erst später auf!(genauer: ab Slide3)}\\
Dieser \alert<2>{Text} ist (nur) auf Slide 2 hervorgehoben!\\
\textcolor<-1>{red}{Rot auf Slide1, Standardfarbe sonst}\\
\begin{itemize}[<+->] %Default-Overlay-Spezifikation (+: interner counter, -: bis)
\item Spiegelstriche %äquivalent:\item<+->...
\item tauchen %äquivalent:\item<+->...
\item nacheinander auf! %äquivalent:\item<+->...
\item<2> Verhalten kann aber auch überschrieben werden!
\end{itemize}
\newcommand<>{\blau}[1]{\textcolor#2{blue}{#1}}
\blau<1,3>{Dieser Text ist auf Slides 1 \& 3 blau!}
\end{frame}
\end{comment}
% -----------

\section{Main Content}

\begin{frame}{A Slide with Math}
\[
E = mc^2
\]
\end{frame}

\begin{frame}{A Slide with Columns}
\begin{columns}
  \column{0.5\textwidth}
    \begin{itemize}
      \item Point A
      \item Point B
    \end{itemize}
  \column{0.5\textwidth}
    \includegraphics[width=\linewidth]{example-image}
\end{columns}
\end{frame}

% -----------

\section{Conclusion}

\begin{frame}{Conclusion}
\begin{itemize}
  \item Summary of results
  \item Take-home message
  \item Outlook
\end{itemize}
\end{frame}

% -----------

\begin{frame}
  \centering
  \Huge Thank you for your attention!
\end{frame}

\end{document}
