% LaTeX-Vorlage zur Erstellung einer Projektarbeit (Dokumentation eines Projekts)
% auf Basis der Vorlage für eine
% Abschlussarbeit in der Fakultät Elektrotechnik, Medien und Informatik an der OTH Amberg-Weiden
% Diese Vorlage entstand im Rahmen des Kurses "LaTeX fürs Studium"
% Aktuelle Version: v0.02
% Stand: 11.01.2026
%
% Changelog:
%
% v0.02: 11.01.2026, Überarbeitung und Aktualisierung der Pakete & Optionen sowie der Eingabe persönlicher Angaben
% v0.01: 03.06.2023, Anpassung der Vorlage für Projektarbeiten (article statt report, keine Titelblätter)
%
\documentclass[12pt, oneside]{article}
\usepackage[T1]{fontenc}
\usepackage[ngerman]{babel}
\usepackage{parskip}  % Absätze: Abstand statt Einrueckung
\usepackage[a4paper,			      % Papierformat A4
            left=2.5cm,				  % Linker Rand
            right=2.5cm,			  % Rechter Rand
            top=1.5cm,				  % Oberer Rand
            bottom=1.5cm,			  % Unter Rand
            marginparsep=5mm,	  % Abstand der Randnotizen
            marginparwidth=10mm,% Breite der Randnotizen
            headheight=7mm,			% Höhe der Kopfzeile
            headsep=1.2cm,			% Abstand der Kopfzeile
            footskip=1.5cm,			% Abstand der Fußzeile
            includeheadfoot]{geometry}

% PDF-Einstellungen
\usepackage[bookmarks,
            raiselinks,
            pageanchor,
            hyperindex,
            colorlinks,
            citecolor=black,
            linkcolor=black,
            urlcolor=black,
            filecolor=black,
            menucolor=black]{hyperref}
\usepackage{xspace}
% Custom commands
\makeatletter
\newcommand*{\group}[1]{\def\@group{#1}}\group{}
\NewDocumentCommand{\insertgroup}{D(){} D[]{}}{%
  \ifthenelse{\equal{\@group}{}}{}{#1\@group#2}%
}
\newcommand*{\insertauthor}{{\def\and{\unskip~\&\xspace}\@author}}
\newcommand*{\inserttitle}{\@title}
\newcommand*{\courseofstudy}[1]{\def\insertcourseofstudy{#1}}
\newcommand*{\keywords}[1]{\def\insertkeywords{#1}}
\newcommand*{\firstexaminer}[1]{\def\insertfirstexaminer{#1}}
\newcommand*{\insertgroupandauthors}{%
  \insertgroup[:~]\insertauthor%
}
\AtBeginDocument{
  \def\and{ \& }
  \hypersetup{pdftitle={\@title},
              pdfauthor={\@author},
              pdfsubject={\@group},
              pdfkeywords={\insertkeywords},
  }
}
\makeatother

\usepackage{fancyhdr}	                % Eigene Kopf- und Fußzeilen
\pagestyle{fancy}                     % Seitenstil 'fancy' aktivieren
\fancyhf{}						                % Vorhandene Einstellungen löschen
\setlength{\headwidth}{\textwidth}    % Kopf- und Fußzeile so breit wie der Haupttext
\fancyfoot[R]{\thepage}               % Seitenzahlen in der Fußzeile rechts
\fancyfoot[L]{\inserttitle}           % Titel in der Fußzeile links
\fancyhead[L]{\insertgroupandauthors} % Gruppe und Autorennamen in der Kopfzeile links
\renewcommand{\sectionmark}[1]{       % Definition der Ausgabe des Abschnitts
  \markboth{Abschnitt \thesection. #1}{}}
\renewcommand{\headrulewidth}{0.5pt}  % Trennlinie zwischen Kopfzeile und Haupttext
\renewcommand{\footrulewidth}{0.5pt}  % Trennlinie zwischen Haupttext und Fußzeile
\fancypagestyle{plain}{				     	  % Anpassung des Seitenstils 'plain' bei Beginn neuer Abschnitte
  \fancyhf{}								          % Vorbelegung löschen
  \fancyfoot[C]{\thepage}				      % Seitenzeilen in der Fußzeile mittig
}

% Mathematischer Formelsatz
\usepackage[intlimits]{amsmath} % Europäische Integrallimites
\usepackage{amssymb}
\usepackage{amsfonts}

% Tabellen
\usepackage{array}
\usepackage{multirow}
\usepackage{booktabs}

% Verschiedenes
\usepackage[largesc]{newpx}	        % Schriftart Palatino für Haupttext & Mathematikmodus
\usepackage{graphicx}							  % Einbinden von Graphiken
\usepackage{tikz}								    % Erstellen von Graphiken + Laden von Standard-Bibliotheken
\usetikzlibrary{arrows.meta, calc, fit, positioning, shapes.geometric}
\usepackage[svgnames,cmyk]{xcolor} 	% Verwendung von Farben
\usepackage[shortlabels, inline]{enumitem}  % Anpassung von Auflistungen/Aufzählungen
\usepackage[autostyle=true]{csquotes} % \enquote{} für passende Anführungszeichen
\usepackage[ngerman, capitalize, nameinlink]{cleveref} % Automatisches Referenzieren
\usepackage{subcaption} % Unterabbildungen/-Tabellen
\usepackage[defaultlines=2, all]{nowidow} % Hurenkinder und Schusterjungen vermeiden
\usepackage{placeins} % Mehr Kontrolle über Gleitobjekte

% SI-Einheiten & Zahlendarstellung
\usepackage[output-decimal-marker={,},  % Komma statt Punkt als Dezimaltrennzeichen
            group-separator={\,},       % Halbes Leerzeichen als Tausendertrennzeichen
            uncertainty-mode=separate,  % ± statt ()
            retain-zero-uncertainty=true,
           ]{siunitx}  % 123 456 789,01234 ± 123

% Zitationen
\usepackage[backend=biber,
            style=ieee,
            sorting=none,
            sortcites=true,
            date=iso,
            seconds=true]{biblatex}
\addbibresource{quellen.bib}


% Darstellung und Formatierung von Quellcode
\usepackage{minted}
\setminted{
  autogobble,
  breaklines,
  linenos,
  numbers=left,
}
% Alternative: Listings
% \usepackage{listings} % Darstellung von Quellcode
% \lstloadlanguages{[ISO]C++,Java,XML,[LaTeX]TeX}
% \lstset{language=C++,
% 	numbers = none,
% 	basicstyle = \small\ttfamily,
% 	keywordstyle = \color{blue},
% 	commentstyle = \color{battleshipgrey},
% 	stringstyle = \color{cardinal},
% 	columns = flexible,
% 	showstringspaces = false
% }


%
% Persönliche Angaben
%
\author{Erika Musterfrau \and Thomas Mustermann}
\group{Gruppe M01} % ggf. leer lassen
\courseofstudy{Medieninformatik}
\title{Projektarbeit Mobile \& Ubiquitous Computing}
\keywords{Android, ESP32, MQTT, Lagesensor, MPU6050, WLAN}
\firstexaminer{Prof.\,Dr. Ulrich Schäfer}

%
% Beginn des Textteils
%

\begin{document}
\thispagestyle{empty}
\begin{center}
  \Large
  Ostbayerische Technische Hochschule Amberg-Weiden\\
  Fakultät Elektrotechnik, Medien und Informatik\\[.8cm]
  \large \insertfirstexaminer\\[.8cm]
  \large Studiengang \insertcourseofstudy\insertgroup(, )\\[.8cm]
  \Large \inserttitle\\[.8cm]
  \large \insertauthor\\[.8cm]
  \today\\[2.5cm]
\end{center}

\vfill
\tableofcontents

\clearpage


\section{Einleitung}

Hier starten Sie mit Ihrer Beschreibung.

Bitte passen Sie in der \LaTeX-Datei \texttt{projektarbeit.tex} am Anfang folgende
Zeilen an:

\begin{listing}[h]
  \inputminted[firstline=143, lastline=151]{latex}{projektarbeit.tex}
\end{listing}

Diese Literaturquellen sollten durch passende ersetzt werden:
\autocite{Breymann:2020,Schaefer_Spurk:2010,ct_LaTeX:2005,paho_android:2023}.

Sie können die Dokumentstruktur ändern. Dies ist lediglich ein Vorschlag.

Bei \href{https://www.heise.de/download/blog/Einfuehrung-in-LaTeX-3599742}{Heise (dies ist ein Link)} finden Sie
eine Zusammenstellung, wie Sie eine \LaTeX-Entwicklungsumgebung installieren, falls Sie nicht
\href{https://www.overleaf.com}{Overleaf} benutzen wollen.

Alternativ gibt es Hephaistos, ein Devcontainer-fähiges Docker-Image speziell für das (lokale) Kompilieren von \LaTeX-Dokumenten:
\begin{itemize}[left=4cm]
  \item[Docker] \url{https://hub.docker.com/r/othawemi/hephaistos}
  \item[Container-Source] \url{https://git.oth-aw.de/latex-tools/hephaistos}
  \item[VS-Code-Devcontainer] \url{https://git.oth-aw.de/latex-tools/hephaistos-vscode}
\end{itemize}



\section{Projektplanung und Vorgehen}

\dots\ \dots\ \dots


\section{Implementierung}

Ihr Code muss hier nicht vollständig wiedergegeben werden, aber evtl. interessante Ausschnitte.

Doku s. \url{https://ctan.org/pkg/minted}

Mit \verb|\inputminted[lastline=14]{C++}{sketch.ino}| können auch (Ausschnitte von) externen Quellcode-Dateien eingebunden werden, sodass \emph{copy \& paste} von Quellcode überflüssig wird.

Es folgt ein C++-Codebeispiel:
\begin{listing}[ht]
  \begin{minted}{C++}
    void setup() {
      Serial.begin(115200);
      // gibt etwas aus über die serielle Leitung:
      Serial.println("Hallo ESP");
    }
  \end{minted}
\end{listing}

\FloatBarrier % Forciert die Ausgabe noch ausstehender Floating-Umgebungen

Java geht auch, hier sieht man gleich die Javadoc-Syntax:
\begin{listing}[ht]
  \begin{minted}{Java}
    /**
    * Ein Hallo-Welt-Programm in Java.
    * Dies ist ein Javadoc-Kommentar.
    *
    * @author Ulrich Schäfer
    * @version 1.0
    */
    public class HalloWelt {
    /**
    * Main-Methode
    *
    * @param args Kommandozeilenparameter
    * @return Rückgabewert
    */
      public static int main(String[] args) {
        System.out.println("Hallo Welt!");
      }
    } // Ende der Java-Klasse HalloWelt
  \end{minted}
\end{listing}


Und zu guter Letzt noch ein XML-Beispiel:
\begin{listing}[h]
  \begin{minted}{XML}
    <spitze>
      <!-- Kommentar in XML -->
      <klammer wert="auf"/>
    </spitze>
  \end{minted}
  \caption{Sie sollten die abgedruckten Quellcodes mit treffenden Titeln versehen und bei Bedarf mittels \texttt{\textbackslash{}label} benennen!}
  \label{src:xml}
\end{listing}



\section{Probleme und Diskussion}

Sie können die Überschrift natürlich umbenennen, falls es keine Probleme gibt :-).



\section{Zusammenfassung und Ausblick}

Die Literatur (für den nächsten Abschnitt) tragen Sie in die Datei \texttt{quellen.bib} ein. Diese Datei hat die BiB\TeX-Syntax (vgl. \url{https://ctan.org/pkg/bibtex}; BiB\TeX\ ist bei \href{https://www.heise.de/download/product/texstudio-62955}{\TeX studio} bzw. \href{https://www.overleaf.com}{Overleaf} dabei; selbstverständlich auch bei Hephaistos).

Zur Literaturverwaltung verwenden Sie am besten \href{https://www.zotero.org}{Zotero} oder \href{https://www.jabref.org}{JabRef}.

\clearpage % - bei Bedarf diesen Seitenumbruch entfernen

\phantomsection
\addcontentsline{toc}{section}{Literatur}
\printbibliography


%\include{anhang} % zum Beispiel hier die ChatGPT-Chatprotokolle einbinden (oder als extra-Datei)

\end{document}
