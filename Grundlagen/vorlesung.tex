\documentclass{article}

%PACKAGES
%----------------------------
\usepackage[ngerman]{babel} %Localisation
\usepackage[a4paper,margin=3cm]{geometry} %Geometry for better Layout
\usepackage[T1]{fontenc} %Fontencoding
\usepackage{microtype} %Better Typography (Sperrsatz)
\usepackage{textcomp} %Companion fonts
%\usepackage{lmodern} %Modern font family
\usepackage{mathpazo} %Palatino (Text) und Pazo (Mathe)
%\usepackage[largesc]{newpx} %Better Palatino
\usepackage{ragged2e} %Schönerer Zeilenfall
\usepackage[autostyle=true]{csquotes} %Quotes
\usepackage{fontawesome5} %Icons
\usepackage[skip=5mm, indent=0cm]{parskip} %Zeileneinzug und Absatzabstand
\usepackage[singlespacing]{setspace}
\usepackage[shortlabels]{enumitem} % Enumeration mit Shortlabels
\usepackage[svgnames, table]{xcolor} %Farben
\usepackage{xkcdcolors}
\usepackage[most]{tcolorbox} %Colorboxes
\usepackage[raiselinks, pageanchor, colorlinks=false, hidelinks]{hyperref} %Verlinkungen (hidelinks verbirgt Linkboxen)
\usepackage{bookmark} % PDF bookmarks
\usepackage[nameinlink, ngerman]{cleveref} %Clevere Referenzen (immer nach hyperref)
\usepackage{mwe} %Beispielbilder
\usepackage[backend=biber,style=ieee,backref=true, date=iso, seconds=true]{biblatex} %Zitation
\usepackage{graphicx} %Abbildungen
\usepackage{subcaption} %Unterabbildungen
\usepackage{float} %Precise figure placement (using H)
\usepackage{multicol} %Multicol Layout
\usepackage{svg} %SVG Images
\usepackage{multirow} %Mehrzeilige Tabellenzellen
\usepackage{array} %Erweiterte Tabellenfunktionen
\usepackage{caption} %Caption Anpassungen
\usepackage{booktabs} %Schönere Tabellenlinien
\usepackage{nicematrix} %Noch schönere Tabellen
\usepackage{tabularx} %Tabellen mit fester Breite
\usepackage{amssymb} %Mathe
\usepackage{amsmath} %Mathe
\usepackage{amsfonts} %Mathe
%\usepackage{array} %Mathe
\usepackage{listings} %Quellcode
\usepackage{inconsolata} %Schöne Monospace Schriftart
\usepackage{minted} %Hervorhebung von Quellcode
\usepackage{mdframed} %Rahmen
\usepackage{relsize} %Text skalieren
\usepackage{fancyhdr} %Kopf- und Fußzeilen
\usepackage{lastpage} %Letzte Seite referenzieren
\usepackage{titlesec} %Section Anpassungen
\usepackage{titletoc} %ToC anpassen
\usepackage{tikz} %TikZ für Diagramme
\usetikzlibrary{arrows.meta,positioning,calc}
\usepackage{pgfplots} %PGFPlots für Diagramme
\pgfplotsset{compat=1.18}
\usepackage{siunitx} %Einheiten
\usepackage{pdflscape} %Landscape
%\usepackage[defaultlines=2,all]{nowidow} %Keine Satzfehler
\usepackage{xparse} %Custom Befehle
\usepackage{menukeys} %Tastatursymbole

%CONFIGS
%----------------------------
{
\setlist{nosep}
%\setlist{itemsep=0ex, topsep=0em} %Globale Listeneinstellungen
\setlist{
  itemsep=0pt,
  topsep=0pt,
  parsep=0pt,
  partopsep=0pt
}
}

{
\setstretch{1} %Zeilenabstand
}

{
\bookmarksetup{depth=2} %Bookmarks depth
}

{
\hypersetup{
colorlinks=true,
urlcolor=CornflowerBlue, % \url and \href
linkcolor=Magenta, % internal links
citecolor=orange,
filecolor=purple
}
}

{
%\titlecontents{section} %Custom ToC
%[1.5em]
%{\bfseries}
%{\contentslabel{2em}}
%{}
%{\hfill\contentspage}
}

{
\titleformat{\section}
{\sffamily\bfseries}
{\thesection}{1em}{}

\titleformat{\subsection}
{\sffamily\bfseries}
{\thesubsection}{1em}{}

\titleformat{\subsubsection}
{\sffamily\bfseries}
{\thesubsubsection}{1em}{}
}

{
\captionsetup{
  font=small,
  labelfont=bf,
  labelsep=colon
}
\captionsetup[table]{hypcap=false}
\captionsetup[listing]{hypcap=false}
\captionsetup[figure]{hypcap=false}
}

{
\usemintedstyle{friendly}
\setminted{
  fontsize=\large,
  linenos=false,
  breaklines=true
}
}

\addbibresource{example.bib}
\addbibresource{quellen.bib}

\title{Super toller Titel}
\author{Your name here}

%DOCUMENT
%----------------------------
\begin{document}

\maketitle
\tableofcontents

\section*{Formatierung \& Umgebung}
{
Heute ist der \today.

\begin{enumerate}[(a)]
\item Erster Punkt in erster Ebene
\item Zweiter Punkt
\begin{enumerate}
\item Erster Punkt in zweiter Ebene
\begin{enumerate}
\item Erster Punkt in dritter Ebene
\item Wieder normal
\end{enumerate}
\end{enumerate}
\item[\faCog] Geändertes Aufzählungszeichen
\item Zweiter Punkt in zweiter Ebene
\item[\faCrown] Dritter Punkt in erster Ebene
\end{enumerate}

\marginpar{Randnotiz}

Hallo~\footnote{Hello Footnote}

\enquote{deutsche \enquote{Anführungszeichen}}

\begin{itemize}
\item \faIcon{windows}
\item \faIcon{linux}
\item[\faIcon{android}]
\item[{\faCheckSquare[regular]}]
\end{itemize}

{\bfseries Fettgedruckt und \textmd{wieder normal}, in einer {\mdseries neuen
\TeX-Gruppe kann dann \textbf{wiederum hin- und} hergeschaltet} werden.}\\
Also\bfseries{} eigentlich\mdseries{} alles \textbf{ganz} \textmd{einfach}, oder?

Small caps (Kapitälchen): \textsc{Hello World}

\begin{itemize}
\item \href{https://github.com}{\faGithub\ GitHub}
\item \url{https://ctan.org/}
\end{itemize}
}

\section*{Formatierung \& Umgebung}
{
\textcolor{red}{Rot}
\textcolor{blue}{Blau}

\textcolor{blue}{\textcolor{cyan}{\textcolor{green}{\textcolor{yellow}{\textcolor{orange}{\textcolor{red}{Re}gen}bo}gen}lei}ter} %rot,orange,gelb,grün,hellblau,blau

{
\color{red}
\colorlet{mycolor}{.}

\textcolor{mycolor}{Meine Farbe}
}

\section{Eine erste Überschrift}\label{ch:first}
Wir befinden uns in Kapitel~\ref{ch:first}, welches
auf Seite~\pageref{ch:first} abgedruckt ist.

\section{Mysection}\label{s:mys}

In \cref{s:mys} auf \cpageref{s:mys} haben wir Farben geübt.

\section{\texorpdfstring{$E=mc^2$}{E = mc2}}
\label{s:einstein}

\section{Hallo}
\label{s:hello}

\begin{figure}
  \centering
  \includegraphics[width=.5\linewidth]{example-image}
  \caption{An example picture}
  \label{fig:example}
\end{figure}

Ein Link zum Beispielbild~\cref{fig:example}. Mehrere Referenzen:~\cref{s:mys,s:hello} oder \crefrange{s:mys}{s:hello}

\section*{Zitieren und Quellenangabe}
\pdfbookmark[1]{Zitieren und Quellenangabe}{sec:quellen}

Hier will ich zitieren~\cite[vgl.][ff.]{Vaswani2017}.
Hier will ich zitieren~\parencite[][]{Vaswani2017}.
Hier will ich zitieren~\autocite[][]{Vaswani2017}.

Jahr und Autorennamen: \citeyear{Vaswani2017} and \citeauthor{Vaswani2017}.

Im Text \autocite{morrisPasswordSecurityCase1979} wird die Passwortsicherheit besprochen.

\citeauthor{kafkaVerwandlung} schrieb \citetitle{kafkaVerwandlung}.


}


\section*{Abbildungen, Tabellen, Quellcode, Formeln}
{

\subsection*{Abbildungen}

\begin{figure}[H]
\centering
\includegraphics[origin=c,angle=45,width=.5\linewidth]{example-image}
  \caption{A new example picture}
  \label{fig:new_example}
\end{figure}

In~\Cref{fig:new_example} sieht man ein Beispielbild.

%Skalierter Text: \scalebox{2}[3]{Hello}\\
%Resize: \resizebox{!}{2cm}{Mein Text}


\subsection*{Tabellen}

Tabelle mit Table-Umgebung (Float):

\begin{table}[H]
\centering
\caption{Example table}
\label{tab:ex}
\begin{tabular}{cc}
A & B \\
1 & 2
\end{tabular}
\end{table}

Tabellen mit Booktabs:

\begin{table}[H]
\centering
\begin{tabular}{ccc}\toprule
A & B & C\\\midrule
1 & 2 & 3\\\bottomrule
\end{tabular}
\caption{Booktabs Tabelle}
\label{tab:booktabs}
\end{table}

Noch schöner mit Nicetable:

\begin{NiceTabular}{ccc}
A & B & C \\
1 & 2 & 3 \\
4 & 5 & 6
\end{NiceTabular}

Mit tabularx auch Tabellen mit Stretch-Breite:

\begin{tabularx}{\linewidth}{lXr}
Left & This column automatically stretches to fill the line width & Right \\
\end{tabularx}

\begin{tabularx}{\linewidth}{lXr}
\toprule
Item & Description & Value \\
\midrule
A & Automatically wrapped description text & 12 \\
B & Short text & 7 \\
\bottomrule
\end{tabularx}

Framed:\\

\begin{mdframed}[
  linewidth=1pt,
  linecolor=blue,
  backgroundcolor=blue!5,
  roundcorner=5pt
]
Text
\end{mdframed}

Relsize:

Normal \relsize{1}Bigger \relsize{-1}Back to normal

\subsection*{Quellcode}

\begin{lstlisting}[escapeinside={<<}{>>}]
int n = 10; <<\textbf{important}>>
double x = <<$\sqrt{2}$>>;
\end{lstlisting}
}

\subsection*{Mathematikmodus}

Limits:
$\sum\limits_{i=1}^n a_i$

$\sum_{i=1}^n a_i$

\[
\vec{a} = \begin{pmatrix}1\\2\end{pmatrix}
\]

\renewcommand{\vec}[1]{\mathbf{#1}}

\[
\vec{a} = \begin{pmatrix}1\\2\end{pmatrix}
\]

\begin{align*}
\operatorname{LReLU}_\alpha(x) &= \max\{\alpha x, x\} \\
\theta &= \operatorname*{argmin}_\theta \mathcal{L}_\theta(x)\\
\theta &= \operatorname{argmin}_\theta \mathcal{L}_\theta(x)\\
\end{align*}

\section*{Befehle, Längen, Zähler, Layout, Metadaten}
{
\subsection*{Befehle}

{
\renewcommand{\arraystretch}{1.3}
\begin{tabular}{cc}
A & B \\
C & D
\end{tabular}
}

\newcommand{\spruch}[1]{\textit{\enquote*{#1}}}
\spruch{Hallo, Welt!}

\newenvironment{myquote}[2][teal]{%
\textbf{\textsc{#2}:}\par
\begin{quote}\color{#1}%
}{\end{quote}}

\begin{myquote}[purple]{Albert Einstein}
Das Leben ist wie Fahrrad fahren. Um die Balance zu halten, musst du in Bewegung bleiben.
\end{myquote}

\subsection*{Längen}

\newlength{\mylen}
\setlength{\mylen}{2cm}

This rule is 2cm long:
\rule{\mylen}{0.4pt}

\subsection*{Zähler (Teil 1)}

Section \arabic{section}.\arabic{subsection}

Page \arabic{page}\\
Page \thepage\\
%Last Page \pageref{LastPage}

\begin{enumerate}
\item First
\item Second (number = \arabic{enumi})
\end{enumerate}

Newcounter:

\newcounter{mycounter}[subsection]

\setcounter{mycounter}{5}
Zähler hier: \arabic{mycounter}

\subsection*{Zähler (Teil 2)}

Zähler da: \arabic{mycounter}
\renewcommand{\themycounter}{\arabic{mycounter}}
Zähler da: \themycounter


\subsection*{Übungsaufgabe}

\NewDocumentCommand{\othermesswert}{ O{1} O{2} }{%
#1 & #2 \\
}

\newcounter{messcounter}

\newcommand{\messwert}[2]{%
\stepcounter{messcounter}%
\arabic{messcounter} & #1 & #2\\%
}

\newenvironment{messtabelle}[1]{%
\setcounter{messcounter}{0}
\begin{tabular}{ccc}\toprule
\multicolumn{3}{c}{\bfseries #1} \\
\# & Zeit [ms] & Spannung [V]\\\midrule
}{%
\bottomrule
\end{tabular}%
}

\begin{messtabelle}{Wichtige Messwerte}
%\othermesswert
\messwert{0.0}{0.000}
\messwert{0.2}{0.011}
\messwert{0.4}{0.234}
\messwert{0.6}{0.723}
\messwert{0.8}{1.022}
\messwert{1.0}{1.120}
\messwert{1.2}{1.143}
\messwert{1.4}{1.152}
\messwert{5.0}{1.161}
\messwert{10.0}{1.159}
\end{messtabelle}

\begin{messtabelle}{Wichtige Messwerte}
\messwert{0.0}{0.000}
\messwert{0.2}{0.011}
\end{messtabelle}

\subsection*{Minipage}

Text before
\parbox[t]{4cm}{
This is a paragraph box.
It wraps text automatically.
}
text after.


\begin{minipage}[t]{0.3\textwidth} \blindtext \end{minipage}\hfill% Zeilenumbruch
\begin{minipage}[c]{0.3\textwidth} \blindtext \end{minipage}\hfill% verhindern,
\fbox{\parbox[b]{0.3\textwidth}{\blindtext}} % aber den Zwischenraum auffüllen.

\subsection*{Metadaten}

Package Hyperref für Metadateneinbindung.

\verb|\hypersetup{...}| in Präambel.
}

\section*{Präsis und Zeichnungen}
{
\subsection*{Präsis}

Siehe \verb|beamer.tex|\dots

\begin{comment}
\subsection*{Zeichnungen}

TikZ ist kein Zeichenprogramm.

\begin{figure}[H]
\begin{tikzpicture}
\draw [xkcdGolden, ultra thick] (0, 0) circle (5mm);
\end{tikzpicture}
\caption{Ein Ring, sie zu knechten, sie alle zu finden.}
\end{figure}

\begin{figure}[H]
\begin{tikzpicture}
\path [draw=blue, thick] (0, 0)-- ++(0.5, 0)--
++(0, 1)-- ++(35:1.221)--
++(-35:1.221)-- ++(0,-1)--
++(0.5, 0);
\path [draw=black, fill=yellow!30]
(1.25, 0.15) rectangle +(0.75, 0.33);
\draw (0.75, 0)-- ++(0, 0.5)-- ++(0.25, 0) |- cycle
(0.8, 0.3) circle (0.025);
\draw [step=0.25, shift={(1, 0.75)}] grid ++(1, 0.5);
\path [draw=purple, <-] (0, 0) to[out=-30, in=-150] (3, 0);
\end{tikzpicture}
\end{figure}

\begin{tikzpicture}
\draw[->] (0,0) -- (3,0);
\foreach \n in {1,2,3,4,5} {
  \draw (\n/2,0) -- ++(0,-0.2) node[below] {\n};
}

\draw[->] (0,-2) -- (3,-2);
\foreach \n/\t in {1/1,2/2,2.72/e,3.14/\pi,4/4} {
  \draw (\n/2,-2) -- ++(0,-0.2) node[below] {$\t$};
}
\end{tikzpicture}


\begin{tikzpicture}
\coordinate (a0) at (-0.15,-4);
\foreach \t [count=\i from 1, remember=\i as \j initially 0]
  in {I,like,trains,!}
{
  \node[draw,circle,right=0.15cm of a\j] (a\i) {\t};
}
\end{tikzpicture}

PGFPlots.

\begin{tikzpicture}
\begin{axis}[
  width=10cm,
  height=6cm,
  % Axis labels
  xlabel={$x$},
  ylabel={$f(x)$},
  % Title
  title={Example Function Plot},
  % Axis limits
  xmin=0, xmax=5,
  ymin=0, ymax=30,
  % Grid
  grid=both,
  major grid style={line width=.2pt, draw=gray!50},
  minor grid style={line width=.1pt, draw=gray!20},
  % Ticks
  tick align=outside,
  tick pos=left,
  xtick distance=1,
  ytick distance=5,
  % Legend
  legend style={
    at={(0.02,0.98)},
    anchor=north west,
    draw=none,
    fill=white,
    font=\small
  },
  % Lines
  every axis plot/.style={
    thick
  }
]

% First plot
\addplot[
  blue,
  domain=0:5,
  samples=100
]
{x^2};

\addlegendentry{$x^2$}

% Second plot
\addplot[
  red,
  dashed,
  domain=0:5,
  samples=100
]
{5*x};

\addlegendentry{$5x$}

\end{axis}
\end{tikzpicture}
\end{comment}

\section*{Der Weisheit letzter Schluss}
{

Mit Steuerung und Klick im PDF geht es zum Editor.

Mit \keys{Strg + Alt + J} geht es vom Code zum PDF.

Dies ist eine normale Hochformat-Seite.

\begin{comment}
\begin{landscape}
Diese Seite ist im Querformat.

\begin{tabular}{|c|c|c|c|}
\hline
A & B & C & D \\
\hline
1 & 2 & 3 & 4 \\
\hline
\end{tabular}
\end{landscape}

Wieder Hochformat.
\end{comment}

Units:

Eine Länge von \SI{12,5}{\meter}
Eine Zeit von \SI{3}{\second}
Eine Geschwindigkeit von \SI{5}{\meter\per\second}

\bigskip

Zahlen sauber ausgerichtet:

\begin{tabular}{S}
\toprule
{Messwert} \\
\midrule
12.3 \\
4.56 \\
123.45 \\
\bottomrule
\end{tabular}


}

\section{Advanced}
{
{
\makeatletter
% @ allowed only here
\makeatother
}

\makeatletter
\typeout{currentlabel = \@currentlabel} %Console log
\makeatother

\makeatletter
\newcommand{\ifsomelabel}{%
{%
  \ifx\@currentlabel\@empty
    No label found
  \else
    Label found: \@currentlabel
  \fi
}%
}
\makeatother

\newcommand{\ifinsection}{%
  \ifnum\value{section}=0
    Not in a section%
  \else
    In section \thesection%
  \fi
}

\ifinsection

\ifsomelabel

}


\newpage

\printbibliography[title=Funny]


\end{document}